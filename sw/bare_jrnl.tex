
\documentclass[journal]{IEEEtran}
\usepackage{natbib}
\usepackage{CJK}

\begin{document}
\title{ Translation approaches for Cross-Language Information
Retrieval (CLIR)}

\author{Mirco Kocher, Student Number: 09-113-739\\Student in Master of Science in
Computer Science of the Universities of Bern, Neuchâtel and Fribourg}

% The paper headers
\markboth{Seminar Workshop - Spring Semester 2014}%
{Seminar Workshop - Spring Semester 2014}

% make the title area
\maketitle


\begin{abstract}
This research paper presents and evaluates the issue of providing
systems for CLIR. Different translation approaches are first elaborated and then compared.
The two studied issues elaborate what and how to translate.
\end{abstract}


\section{Introduction}
\IEEEPARstart{P}{eople} may write a query in one language and understand answers given in another. This is for instance when regarding very short text in Question and Answer format or just factual information for travel. Many documents contain non-textual information such as images, videos and statistics that can be understood regardless of the language involved and do not need translation.

Next to the two most common working languages in the European Union English and French there are 22 other official languages. While the EU encourages all its citizens to be able to speak two languages in addition to their mother tongue many  are not bilingual. \cite{ebs386} Others can read documents written in another language but cannot formulate a query to search it, or at least cannot provide reliable search terms in a form comparable to those found in the documents being searched. The challenge is "given a query in any medium and any language, select relevant items from a multilingual multimedia collection which can be in any medium and any language, and present them in the style or oder most likely to be useful to the querier, with identical or near identical objects in differen media or languages appropriately identified." \cite{oard97}

For the withing-language retrieveal the implementation is essentially separated into two phases, namely an indexing and a matching phase. Such a system first indexes the documents offline in advance and then in the second step reacts online to the users query. The created index allows to look up features from the query and calculate a score for each matching document. This is faster than searching a large dataset at query execution time with a linear scan and results in an efficient system with effective retrieval. The system builds an inverted index (like a hash table) that allows efficient look-up of a feature and returns a list of all documents containing the given feature. The index is said to be inverted because each feature is associated with a pair containing a document number and the corresponding frequency that denotes how often this term occurs in this document. In the matching phase the system performs {\it n} look-ups for a query with {\it n} different features. All documents that are not included in the union of these {\it n} lists receive a score of 0 and won't be considered further. For all other documents, the similarity scores are calculated according to the number of features they contain. There are different algorithms to calculate a similarity score that are based on the idea that docuemts and queries are vectors in a high-dimensional space. The coefficients represent features with a weighting (like the tf-idf\footnote{term frequency in the document multiplied with the inverse document frequency, that is the number of documnets containing this term} or the cosine\footnote{the tf-idf normalized by its length} model) and binary vector operations (like the inner product\footnote{the sum of the products of the corresponding entries of the two vectors}, dice\footnote{two times the inner product divided by the sum of the length of each vector} or cosine\footnote{the inner product divided by the product of the root of the length of each vectors} formula) are used for the calculation of the similarity score. In the end the system returns a ranked list of documents with descending similarity scores.

To retrieve documents across languages, that is written in languages different from the language used for query formulation, the classic information retrieval mechanisms have to be extended by Cross-Lanugage Infromation Retrieval systems. There is a difference between what they translate (such as the query or document only or a combination of both) and how they translate (either using machine-readable dictionaries, with machine translation or applying a statistical approach). This system manages a language mismatch between query and parts of the document collection where either:
\begin{itemize}
	\item the document collection is monolingual, but the users can formulate queries in a different language.
	\item the document collection contains documents in multiple languages and users can query the entire collection in any language.
	\item the document collection contains documents with mixed-language content and users can query the entire collection in any language.
\end{itemize}
A Multilingual Infromation Retrieval (MLIR) system covers all the above cases plus the basic withing-language retrieval. There are four choices for crossing the language gap between query and documents. We can either:
\begin{enumerate}
	\item translate the query into the language of the documents
	\item translate the documents into the language of the query
	\item translate both the query and the documents into an intermediary language
	\item translate nothing
\end{enumerate}
There are direct advantages and disadvantages to all options. With the second choice the whole corpus has to be translated which uses more storage space with each covered language and is a time-consuming process. With improving translation systems the whole document collection has to be preiodically re-translated to take advantage of these improvements. However the whole translation process can be shifted to the offline portion and avoids any speed penalty at retrieval time. Also the context of terms is available and helps disambiuate words with multiple meanings. On the contrary in the first choice only the words in the query (which is usually short) are translated and avoids these problems. However, since user queries tend to be short and thus offer little context to handle ambiguous terms. The third choice can be used if there is no direct translation available or the quality is poor and the extra translation results in a better retrieval. For similar languages such as in the Nordic languages Danish, Swedish and Norwegian the query might not need to be translated based on the similar vocabulary and with a spelling correction algorithm one language can be seen as a mis-spelled form of another.

Before applying any translation method the text in question has to be preprocessed. In general the text is transformed to lowercase to improve matching regardless of the capitalization (for instance when the word is at the beginning of a sentence). Compound words that don't exist in the target language have to be segmentated and on the other hand tokens have to be compounded to represent a meaningful word. The German word "Bundesbankpr\"{a}sident" should be decoumpounded to "Bund" + es + "Bank" + "Pr\"{a}sident" which is then translated to "federal bank CEO". Conversely when translating the Chinese word \begin{CJK}{UTF8}{gbsn}中国人\end{CJK} to the English language the three logograms when segmeted mean "middle" "kingdom" and "people" which should be compounded and translate to "Chinese".\cite{ir13}
Additionally the text is modified using a stemmer which conflates different token of the same word type. For instance the singular and plural form (like "horse" and "horses") or different grammatical cases (such as the English noun "Prague" in the Czech language where the dative form is "Praze" and the genitive form is "Prahy" are merged with the nominative form "Praha" \footnote{People sometimes use "Praha" in English instead of Prague but mostly forget to decline it. It would obviously be "to be in Praze" and "go to Prahy".}).
Sometimes a stopword list is applied to remove frequent and insignificant terms with the goal to reduce the size of the inverted file. Such a list may contain only one term ("the") as in the WIN system (Thomson Reuters), nine terms ("an", "and", "by", "for", "from", "of", "the", "to", "with") as suggested by the DIALOG system or may like the SMART system include 571 words (e.g. "a", "all", "are", "is", "it", "just", "while", "who", "with", ...).
As a consequence, the length of a piece of text in the source language, and the length of its representation in the target language may differ widely.

There are some problems that come with the translation. One of the most prominent problem is the insufficient lexical coverage where some words have no translation such as for abbreviations or names. This can be countered by using a specialized thesauri with names of persons ("Gorbachev" in English and "Gorbatschow" in German), arts ("Mona Lisa" in English is "La Gioconda" in Italian) and cities ("Lisbon" in English is "Lisboa" in Portuguese) and a dictionary for codes ("WHO" in English is "OMS" in French).
Depending on the previously applied stopword list there might have been new problems introduced. The query "vitamin A" is transformed to "vitamin" when using the SMART system. In this case any document refering to any vitamin is retrieved even if it is about the vitamin C that is of no use for the querier. The same problems appear for the queries "IT engineer" and "WHO goals".
Assuming the systems applies a stopword list that preserves the word "who" but uses a dictionary of codes. In this case the query "Who won the Tour de France in 1995" could be translated to "OMS gagn\'{e} le tour de France en 1995".
Another prominent problem in all CLIR systems is the translation ambiguity. Each word-by-word translation using a machine-readable dictionary returns more possible expressions for each individual term. By simply using all available traslations, the number of terms in the destination language that is substituted for each term in the source language can vary widely. Assuming we use the Merriam-Webster Spanish Online dictionary to replace each word with all given translation alternatives. The Spanish query "Contrabando de Meterial Radioactivo" is transformed (when ignoring the Spanish stopword "de") to "smugglig, contraband; material, physical, real, equipment, gear; radioactive". The resulting query now contains five different terms for the word "Material" which will influence the document retrieval.
%variety in quality of the available resources.

\section{State of the Art}

Most important previous work

Summarize them

Limit of current practices



\section{Formalism}
Notation used

Example



\section{Evaluation}
Benchmarks

Interpretation



\section{Conclusion}
Recap main idea

Main results found

Improvements/applications

\cite{gollins01}
\cite{peters12}
\cite{savoy09}
\cite{yu09}

\bibliographystyle{unsrt}
\bibliography{ref}

\end{document}
