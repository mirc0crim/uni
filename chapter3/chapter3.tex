\documentclass[10pt,a4paper]{beamer}
\usetheme{Singapore}
\setbeamercovered{transparent}
\beamertemplatenavigationsymbolsempty % Comment this out if you want a navigation bar

\usepackage[utf8]{inputenc}
\usepackage[english]{babel}

%math packages
\usepackage{amsmath}
\usepackage{amsfonts}
\usepackage{amssymb}

%package for rule inferences
\usepackage{bussproofs}

%package for handling verbatim
\usepackage{fancyvrb}

% packages for drawing
\usepackage{pgf}
\usepackage{tikz}
\usetikzlibrary{arrows,matrix}

% package for strikethrough text
\usepackage{ulem}

\logo{
	\includegraphics[height=1cm]{coq_logo.png}
}

% uncover everything in a step-wise fashion
%\beamerdefaultoverlayspecification{<+->}

\usepackage{color}
\usepackage{xcolor}

\definecolor{coqGreen}{rgb}{0,0.5,0}

% Environment for user input / source code
\DefineVerbatimEnvironment
{user}{Verbatim}
{formatcom=\color{coqGreen}}

% Environment for Coq output
\DefineVerbatimEnvironment
{coq}{Verbatim}
{formatcom=\color{blue}}

% Customizable emphasis
\providecommand{\highlight}[1]{\textcolor{red}{#1}}
\providecommand{\Highlight}[1]{\textcolor{red}{\textbf{#1}}}
\title{Coq Proof Assistant: Propositions and Proofs}
\author{Mirco Kocher}
\institute{
	Logic and Theory Group\\
	Institute of Computer Science and Applied Mathematics\\
	Universität Bern
}
\date{2012}
\begin{document}

{ \logo{} % Suppress logo for title page
\begin{frame} % Title page
	\begin{center}
		\includegraphics{coq_logo.png}
	\end{center}	
	\titlepage	
\end{frame}
}

\begin{frame}{Overview}
\tableofcontents[pausesections, pausesubsections]
\end{frame}

\section{Minimal Propositional Logic}

\subsection{Basics}
\begin{frame}{Propositional logic}
	$(P \Rightarrow Q) \Rightarrow ((Q \Rightarrow R) \Rightarrow (P \Rightarrow R))$
	\begin{itemize}
		\item $P$, $Q$, $R$ are propositions
		\pause
		\item Goal is to prove the implication
	\end{itemize}
\end{frame}
\begin{frame}{Tarski approach}
	$(P \Rightarrow Q)$
	\begin{itemize}
		\item Question "is the proposition P true?"
		\pause
		\item Classical logic
		\pause
		\item Assign to every variable a denotation true or false
		\pause
		\item Truth table
		\pause		
		\item Formula is valid iff true in all cases
	\end{itemize}
\end{frame}
\begin{frame}{Heyting approach}
	$(P \Rightarrow Q)$
	\begin{itemize}
		\item Question "what are the proofs of P (if any)?"
		\pause
		\item Intuitionistic logic
		\pause
		\item Obtain a proof of Q from a proof of P
		\pause		
		\item Arbitrary proof of P constructs a proof of Q
		\pause
		\item Coq system follows this approach
	\end{itemize}
\end{frame}
\begin{frame}{Translate from Heyting into Coq}
	$(P \Rightarrow Q)$
	\begin{itemize}
		\item Curry-Howard isomorphism is the direct relationship between computer programs and proofs
		\pause
		\item Proofs are seen as programs
		\item Statements are treated as specifications
		\pause
		\item The "$\Rightarrow$" becomes "$\rightarrow$"
	\end{itemize}
	\pause
	$(P \rightarrow Q)$
\end{frame}
\begin{frame}{Overview}
	\begin{itemize}
		\item Present types and tools used to reason in Coq
		\pause
		\item Use tactics to obtain proofs in a semi-automatic way
		\pause
		\item Control the flow of an interactive proof
		\pause				
		\item More details on tactics and corrresponting typing rules
		\pause
		\item Composing tactics, tacticals
		\pause
		\item Proof irrelevance
	\end{itemize}
\end{frame}

\subsection{Definition}
\begin{frame}[fragile]{Sorts}
	Prop is one of the predefined sorts
	\begin{itemize}
		\item Plays a similar role as Set 
		\pause
		\item Set is for programs and specifications (\verb|nat -> nat|, \verb|bool|)\\
		Prop is for proofs and Propositions ($P, Q, R$)
	\end{itemize}
	\pause
	Definition: Every type $P$ whose type is the sort \verb|Prop| is called a proposition. Any term $t$ whose type is a propostion is called a proof term (=proof).
\end{frame}
\begin{frame}[fragile]{Hypothesis}
	\begin{user}
	Hypothesis h:P
	\end{user}
	\begin{itemize}
		\item Local declaration
		\pause
		\item $h$ is the name of the hypothesis
		\pause
		\item P is its statement
		\pause
		\item Synonymous to
		\begin{user}
		Variable h:P
		\end{user}
		\pause
		\item Use
		\begin{user}
		Hypotheses
		\end{user}
		or
		\begin{user}
		Variables
		\end{user}
		to declare several at a time
	\end{itemize}
\end{frame}
\begin{frame}[fragile]{Axiom}
	\begin{user}
	Axiom x:P
	\end{user}
	\begin{itemize}
		\item Global declaration
		\pause
		\item Synonymous to
		\begin{user}
		Parameter x:P
		\end{user}
	\end{itemize}
	\pause
	\medskip
	Environment contains axioms\\
	\pause
	\medskip
	Context contains hypotheses\\
	\pause
	\medskip
	$E, \Gamma \vdash \pi : P$
\end{frame}


\section{Goal-directed proof}

\subsection{Definition}
\begin{frame}{Goals and Tactics}
	A goal is the pairing of two pieces of information: a local context $\Gamma$ and a type $P$ that is well-formed in this context.\\
	\medskip
	\pause
	Goal: $E, \Gamma \vdash^? P$\\
	\pause
	Term $t$ of type $P$ is called a $solution$\\
	\pause
	\bigskip
	Tactics:\\
	Can be applied to a goal.\\
	Decompose this goal into simpler goals and create new goals\\
	\pause
	\medskip
	If $g$ is input goal and $g_1$, $g_2$, ..., $g_k$ are output goals, then it's possible to construct a solution of $g$ from the solutions of goals $g_i$.
\end{frame}
\begin{frame}{Transitivity}
	Proove this implication by introducing tactics as needed
	\begin{center}
		$(P \rightarrow Q) \rightarrow (Q \rightarrow R) \rightarrow (P \rightarrow R)$
	\end{center}
\end{frame}
\begin{frame}[fragile]{Example 1}
	$(P \rightarrow Q) \rightarrow (Q \rightarrow R) \rightarrow (P \rightarrow R)$
	\pause
	\begin{user}
		Section Example1.
		Variables P Q R T : Prop.
	\end{user}
	\pause
	\begin{coq}
		P is assumed
		Q is assumed
		R is assumed
		T is assumed
	\end{coq}
\end{frame}
\begin{frame}[fragile]{Example 1}
	$(P \rightarrow Q) \rightarrow (Q \rightarrow R) \rightarrow (P \rightarrow R)$
	\pause
	\begin{user}
		Theorem imp_trans : (P->Q) -> (Q->R) -> P -> R.
	\end{user}
	\pause
	\begin{coq}
		1 subgoal

		P : Prop
		Q : Prop
		R : Prop
		T : Prop
		===============
		 (P->Q) -> (Q->R) -> P -> R
	\end{coq}
	\pause
	\begin{user}
		Proof.
	\end{user}
\end{frame}

\subsection{Tactics}
\begin{frame}[fragile]{intros}
	Goal: $(P \rightarrow Q) \rightarrow (Q \rightarrow R) \rightarrow (P \rightarrow R)$ \\
	\medskip
	\pause
	\begin{user}
	intros H H' p.
	\end{user}
	\pause
	\begin{itemize}
		\item Transform the task of constructing a proof into proving R with those hypotheses added
		\pause
		\item $H:P \rightarrow Q$
		\pause
		\item $H':Q \rightarrow R$ and
		\pause
		\item $p:P$
		\pause
		\item New subgoal: $R$
	\end{itemize}
	\pause
	Simplifies the statement to prove and increases the resources available
\end{frame}
\begin{frame}[fragile]{Example 1}
	$(P \rightarrow Q) \rightarrow (Q \rightarrow R) \rightarrow (P \rightarrow R)$
	\pause
	\begin{user}
		intros H H' p.
	\end{user}
	\pause
	\begin{coq}
		1 subgoal

		P : Prop
		Q : Prop
		R : Prop
		T : Prop
		H : P->Q
		H' : Q->R
		p : P
		===============
		 R
	\end{coq}
\end{frame}
\begin{frame}[fragile]{apply}
	Subgoal: $R$\\
	\medskip
	Hypothesis $H': Q \rightarrow R$ and $H:P \rightarrow Q$
	\medskip
	\pause
	\begin{user}
	apply H'.
	\end{user}
	\begin{itemize}
		\item Use the hypothesis $H'$ to advance our proof
		\pause
		\item Argument has to be a premise and a conclusion
		\pause
		\item Creates new goal for the premise
		\pause
		\item New subgoal: $Q$
	\end{itemize}
	\pause
	Applying hypothesis H gives the new subgoal $P$
\end{frame}
\begin{frame}[fragile]{Example 1}
	$(P \rightarrow Q) \rightarrow (Q \rightarrow R) \rightarrow (P \rightarrow R)$
	\pause
	\begin{user}
		apply H'.
	\end{user}
	\pause
	\begin{coq}
		1 subgoal

		P : Prop
		Q : Prop
		R : Prop
		T : Prop
		H : P->Q
		H' : Q->R
		p : P
		===============
		 Q
	\end{coq}
\end{frame}
\begin{frame}[fragile]{Example 1}
	$(P \rightarrow Q) \rightarrow (Q \rightarrow R) \rightarrow (P \rightarrow R)$
	\pause
	\begin{user}
		apply H.
	\end{user}
	\pause
	\begin{coq}
		1 subgoal

		P : Prop
		Q : Prop
		R : Prop
		T : Prop
		H : P->Q
		H' : Q->R
		p : P
		===============
		 P
	\end{coq}
\end{frame}
\begin{frame}[fragile]{Tactic assumption}
	Subgoal: $P$\\
	\medskip
	Hypothesis $p: P$
	\medskip
	\pause
	\begin{user}
		assumption.
	\end{user}
	\begin{itemize}
		\item Statement to proof is exactly statement of hypothesis $p$
		\pause
		\item Succeeds without generating any new goal
	\end{itemize}
	\pause
	\medskip
	\begin{coq}
	No more subgoals
	\end{coq}
	\pause
	Proof complete
\end{frame}
\begin{frame}[fragile]{Finish}
	\begin{user}
	Qed.
	\end{user}
	\pause
	\begin{itemize}
		\item Every proof should be terminated by the Qed command
		\pause
		\item Saves the theorem's name, statement and proof term
		\pause
		\item Displays the sequence of tactics.
	\end{itemize}
	\begin{coq}
		intros H H' p.
		apply H'.
		apply H.
		assumption.
	\end{coq}
	\pause
	\medskip
	\begin{user}
		Print imp_trans.
	\end{user}
	\pause
	\begin{itemize}
		\item Shows the proof like any $Gallina$ definition
	\end{itemize}
	\begin{coq}
		imp-trans = fun (H:P->Q)(H':Q->R)(p:P) => H' (H p)
		 : (P->Q) -> (Q->R) -> P -> R
	\end{coq}
\end{frame}
\begin{frame}[fragile]{Reading the proof term}
	\begin{user}
		Print imp_trans.
	\end{user}
	\begin{coq}
		imp-trans = fun (H:P->Q)(H':Q->R)(p:P) => H' (H p)
		 : (P->Q) -> (Q->R) -> P -> R
	\end{coq}
	\pause
	(H) : Assume P $\rightarrow$ Q\\
	(H') : Assume Q $\rightarrow$ R\\
	(p) : Assume P\\
	(1) : By using (p) we get P\\
	(2) : By applyig (H) we get Q\\
	(3) : By applying (H') we get R\\
	\pause
	\medskip
	Discharging H, H' and p, we get $(P \rightarrow Q) \rightarrow (Q \rightarrow R) \rightarrow P \rightarrow R$
\end{frame}

\section{Structure}

\subsection{Typing rules}
\begin{frame}[fragile]{Tactic auto}
	Not all the details for the proof are needed\\
	\medskip
	\pause
	Transitivity is a rather trivial statement \\
	A few automatic tactics are able to solve this goal
	\pause
	\begin{user}
		Theorem transitivity : (P->Q) -> (Q->R) -> P -> R.
		Proof.
	\end{user}
	\pause
	\begin{user}
		auto.
		Qed.
	\end{user}
\end{frame}
\begin{frame}[fragile]{Building Rules}
	Same kind of typing rules that control the construction of proposition as those used to build simple specifications.\\
	\medskip
	\pause
	Difference is in the use of $Prop$ instead of $Set$.\\
	\pause
	\medskip
	$E, \Gamma \vdash id : Prop$\\
	\pause
	\medskip
	\begin{user}
		Check id.
	\end{user}
	\pause
	\begin{coq}
		id : Prop
	\end{coq}
\end{frame}
\begin{frame}[fragile]{General Prod}
	\begin{prooftree}
		\AxiomC{$E,\Gamma \vdash P : Prop$}
		\AxiomC{$E,\Gamma \vdash Q : Prop$}
		\BinaryInfC{$E, \Gamma \vdash P \rightarrow Q : Prop$}\textbf{Prod-Prop}
	\end{prooftree}
	\pause
	As the rules Prod-Set and Prod-Prop only differ in their use of sorts they can be presented as a single rule with parameter sort $s$
	\pause
	\begin{prooftree}
		\AxiomC{$E,\Gamma \vdash A : s$}
		\AxiomC{$E,\Gamma \vdash B : s$}
		\BinaryInfC{$E, \Gamma \vdash A \rightarrow B : s$}\textbf{Prod}
	\end{prooftree}
	\begin{center}
		With $s \in$ \{Set, Prop\}
	\end{center}
\end{frame}
\begin{frame}[fragile]{Var rule}
	\begin{prooftree}
		\AxiomC{$(x:P) \in E \cup \Gamma$}
		\UnaryInfC{$E, \Gamma \vdash x : P$}\textbf{Var}
	\end{prooftree}
	\pause
	This corresponds to the tactic \verb|assumption| or, if $x$ is not in the context, \verb|exact x|.
\end{frame}
\begin{frame}[fragile]{exact}
	The exact tactic takes any term (with right type) as argument\\
	\pause
	\smallskip
	The whole proof could be given as an argument\\
	\pause
	\medskip
	\begin{user}
		Theorem delta : (P->P->Q) -> P -> Q.
		Proof.
	\end{user}
	\pause
	\begin{user}
		exact (fun (H:P->P->Q)(p:P) => H p p).
		Qed.
	\end{user}
	\pause
	\medskip
	Or even shorter:
	\begin{user}
		Theorem delta : (P->P->Q) -> P -> Q.
		Proof (fun (H:P->P->Q)(p:P) => H p p).
	\end{user}
	\pause
	Uses the context hypotheses
\end{frame}
\begin{frame}[fragile]{Modus Ponens}
	Apply tactic uses the Modus Ponens\\
	\pause
	\medskip
	\begin{prooftree}
		\AxiomC{$E,\Gamma \vdash t : P \rightarrow Q$}
		\AxiomC{$E,\Gamma \vdash t' : P$}
		\BinaryInfC{$E, \Gamma \vdash t t' : Q$}\textbf{App}
	\end{prooftree}
	\pause
	Term $t$ : $P_1 \rightarrow P_2 \rightarrow ... \rightarrow P_n \rightarrow Q$ \\
	Goal : $P_k \rightarrow P_{k+1} \rightarrow ... \rightarrow P_n \rightarrow Q$, also called "head type" of rank $k$\\
	The term $Q$ is called "final type" if it's not an arrow type.\\
	Head and final types play a role in later tactics
	\pause
	\medskip
	\begin{user}
		apply t
	\end{user}
	generates k-1 goals with statements $P_1, ..., P_{k-1}$\\
	\pause
	\smallskip
	if new goals have solution $t_1, t_2, ..., t_{k-1}$\\
	solution is $t$ $t_1$ $t_2$ ... $t_{k-1}$ for initial goal
\end{frame}
\begin{frame}[fragile]{Lam rule}
	Proof of $P \rightarrow Q$ is a function mapping any proof of $P$ to a proof of $Q$
	\pause
	\begin{prooftree}
		\AxiomC{$E, \Gamma :: (H : P) \vdash t : Q$}
		\UnaryInfC{$E, \Gamma \vdash$ fun $H : P \Rightarrow t : P \rightarrow Q$}\textbf{Lam}
	\end{prooftree}
	\pause
	$\Gamma \vdash^? P \rightarrow Q$\\
	\medskip
	\pause
	$\Gamma :: (H : P) \vdash^? Q$\\
	\pause
	\medskip
	If term $t$ is a solution to the subgoal then "fun $H : P \Rightarrow t$" is a solution to the initial goal.\\
	\pause
	\medskip
	This corresponds to the tactic intro
\end{frame}
\begin{frame}[fragile]{intros}
	\begin{user}
		intros v_1, v_2, ..., v_n
	\end{user}
	is the same as
	\begin{user}
		intro v_1, intro v_2, ..., intro v_n
	\end{user}
	\pause
	intro $v_1$ takes the first implication as a hypothesis called $v_1$\\
	\medskip
	\pause
	Proof Theorem K : $P \rightarrow Q \rightarrow P$\\
	\pause
	Goal is  $P \rightarrow Q \rightarrow P$\\
	\pause
	\begin{user}
		intro p.
	\end{user}
	Hypothesis $p:P$ and new Goal $Q \rightarrow P$\\
	\pause
	\bigskip
	Without a name intro (resp intros) generates exactly one hypothese (resp as many hypotheses as possible) and automatically names them.
\end{frame}
\begin{frame}[fragile]{Handling}
	\begin{user}
		Show i
	\end{user}
	Display goal $i$ with complete context\\
	Coq displays the current goal after each proof step
	\pause
	\medskip
	\begin{user}
		Undo n
	\end{user}
	Go back $n$ steps and try an alternative if goal can not be solved
	\pause
	\medskip
	\begin{user}
		Focus n
	\end{user}
	Focus the attention on the $n$th subgoal to prove
	\pause
	\medskip
	\begin{user}
		Restart
	\end{user}
	Go back to the beginning of the proof
	\pause
	\medskip
	\begin{user}
		Abort
	\end{user}
	Abandon the proof
\end{frame}

\subsection{Composing tactics}
\begin{frame}[fragile]{Simple composing}
	Combine tactics without stopping at intermediary subgoals\\
	\pause
	\medskip
	Goal: $P \rightarrow Q \rightarrow (P \rightarrow Q \rightarrow R) \rightarrow R$
	\pause
	\begin{user}
		intros p q H; apply H; assumption.
	\end{user}
	\pause
	Like chess: forsee results of tactics\\
	\pause
	\medskip
	If any tatcic fails, then the whole combination fails
\end{frame}
\begin{frame}[fragile]{General composing}
	Tactics can generate multiple subgoals \\
	\pause
	\medskip
	Goal: $(P \rightarrow Q \rightarrow R) \rightarrow (P \rightarrow Q) \rightarrow (P \rightarrow R)$
	\pause
	\begin{user}
		intros H H' p; apply H; [assumption | apply H'; assumption].
	\end{user}
	\pause
	Two subgoals $P$ and $Q$\\
	\pause
	\medskip
	First is solved with assumption\\
	\pause
	Other one first has to apply H' and then use assumption
\end{frame}
\begin{frame}[fragile]{More composing}
	If a tactic fails automatically use another tactic
	\pause
	\begin{user}
		intros H p; apply H; (assumption || intro H').
	\end{user}
	\pause
	If assumption fails, then intro H' is used\\
	\pause
	\bigskip
	A tactic can be left unchanged to finish every subgoal in one go\\
	\pause
	\smallskip
	Goal: $(P \rightarrow Q) \rightarrow (P \rightarrow R) \rightarrow (P \rightarrow Q \rightarrow R \rightarrow T) \rightarrow P \rightarrow T$
	\pause
	\begin{user}
		intros H H0 H1 p.
		apply H1; [idtac | apply H | apply H0]; assumption
	\end{user}
\end{frame}
\begin{frame}[fragile]{Fail}
	Tactic that always fails \\
	\pause
	\medskip
	Goal: $(P \rightarrow Q) \rightarrow (P \rightarrow Q)$
	\pause
	\begin{user}
		intro X; apply X; fail.
	\end{user}
	\pause
	This combination succeeds; there are no more subgoals after "apply X"\\
	\pause
	\medskip
	Goal: $((P \rightarrow P) \rightarrow (Q \rightarrow Q) \rightarrow R) \rightarrow R$
	\begin{user}
		intro X; apply X; fail.
	\end{user}
	\pause
	This combination fails; there are subgoals left after "apply X"
\end{frame}
\begin{frame}[fragile]{Try}
	Combination of tactics that never fail \\
	\pause
	\medskip
	Goal: $(P \rightarrow Q \rightarrow R \rightarrow T) \rightarrow (P \rightarrow Q) \rightarrow (P \rightarrow R \rightarrow T)$
	\pause
	\begin{user}
		intros H H' p r.
		apply H; try assumption; apply H'; assumption.
	\end{user}
	\pause
	"try tac" behaves like "tac $\|$ idtac"\\
	\pause
	\medskip
	tac is either applied or the subgoal is left unchanged
\end{frame}
\begin{frame}[fragile]{Unprovalbe Propositions}
	There are goals with no solution at all\\
	\pause
	\medskip
	Even though they are valid in classical logic\\
	\pause
	\medskip
	Peirce's formula: $(((P \rightarrow Q) \rightarrow P) \rightarrow P)$\\
	\pause
	\medskip
	Truth table shows it is a valid formula
\end{frame}

\begin{frame}[fragile]{Overview}
	Next Week:
	\begin{itemize}
		\item finish typing rules
		\pause
		\item More details on tactics
		\pause
		\item Composing
		\pause
		\item Proof irrelevance
	\end{itemize}
\end{frame}

\end{document}