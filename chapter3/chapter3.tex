\documentclass[10pt,a4paper]{beamer}
\usetheme{Singapore}
\setbeamercovered{transparent}
\beamertemplatenavigationsymbolsempty % Comment this out if you want a navigation bar

\usepackage[utf8]{inputenc}
\usepackage[english]{babel}

%math packages
\usepackage{amsmath}
\usepackage{amsfonts}
\usepackage{amssymb}

%package for rule inferences
\usepackage{bussproofs}

%package for handling verbatim
\usepackage{fancyvrb}

% packages for drawing
\usepackage{pgf}
\usepackage{tikz}
\usetikzlibrary{arrows,matrix}

% package for strikethrough text
\usepackage{ulem}

\logo{
	\includegraphics[height=1cm]{coq_logo.png}
}

% uncover everything in a step-wise fashion
%\beamerdefaultoverlayspecification{<+->}

\usepackage{color}
\usepackage{xcolor}

\definecolor{coqGreen}{rgb}{0,0.5,0}

% Environment for user input / source code
\DefineVerbatimEnvironment
{user}{Verbatim}
{formatcom=\color{coqGreen}}

% Environment for Coq output
\DefineVerbatimEnvironment
{coq}{Verbatim}
{formatcom=\color{blue}}

% Customizable emphasis
\providecommand{\highlight}[1]{\textcolor{red}{#1}}
\providecommand{\Highlight}[1]{\textcolor{red}{\textbf{#1}}}

\title{Coq Proof Assistant: Propositions and Proofs}
\author{Mirco Kocher}
\institute{
	Logic and Theory Group\\
	Institute of Computer Science and Applied Mathematics\\
	Universität Bern
}
\date{2012}

\begin{document}

{ \logo{} % Suppress logo for title page
\begin{frame} % Title page

	\begin{center}
		\includegraphics{coq_logo.png}
	\end{center}
	
	\titlepage
	
\end{frame}
}

\begin{frame}{Frame title goes here}

	Header text of the slide

	\begin{itemize}

	\item First part of the slide.

	\pause		
	
	\item Second part of the slide, to be uncovered later.
	
	\begin{itemize}
		\item It has two sub-items, one immediately visible.

		\pause	
	
		\item And the other uncovered with the next page.
	\end{itemize}		
	
	\pause	
	
	\item Last part of the slide, to be uncovered later still.	
	
	\end{itemize}
	
\end{frame}

\begin{frame}{Emphasis is everything}

	The following word is \highlight{emphasized} is a way that's \highlight{clearly visible} on a beamer.
	In case you want a \highlight{stronger} emphasis, it's \Highlight{possible too}.
	
	\bigskip
	
	Commands used for that are defined in preamble.tex, you can tweak the visual style from one place.

\end{frame}

\begin{frame}{Columns and paragraphs}

	\begin{center}

		It makes sense to center-align text sometimes.		
		
	\end{center}

	\begin{columns}
		
		\begin{column}{0.4\textwidth}
		
			Arranging it in columns is also a possibility.
			
		\end{column}	
		
		\begin{column}{0.6\textwidth}
		
			Note that column width can be custom.		
		
		\end{column}		
		
	\end{columns}
	
	\bigskip
	
	Don't neglect commands for manual spacing: 
	\smallskip
	
	smallskip,
	\medskip
	
	medskip,
	\bigskip	
	
	bigskip.

\end{frame}

% ! IMPORTANT !
% Note the use of [fragile] in \frame
% It is ALWAYS needed if you use any verbatim-like environments or \verb|...|
\begin{frame}[fragile]{Verbatim and Coq environments}

	\begin{verbatim}
		Sometimes you need verbatim text.
		Note: that makes \frame [fragile].
	\end{verbatim}
	
	Preamble defines two color-coded environments for Coq code and output, namely \verb|user| and \verb|coq|:
	
	% Note: fancyvrb ignores tabs - to use indented code, you need to use spaces
	\begin{user}
		Theorem Fermat:
		  forall x y z n : nat, x ^ n + y ^ n = z ^ n -> n <= 2.
		Proof.
		  intros.
	\end{user}
	
	\begin{coq}
		1 subgoal
		x : nat
		y : nat
		z : nat
		n : nat
		H : x ^ n + y ^ n = z ^ n
		______________________________________(1/1)
		n <= 2
	\end{coq}

\end{frame}

\begin{frame}{Inference trees}

	You can use \highlight{bussproofs} to display inference rules and derivations:
	
	\begin{prooftree}	
		\AxiomC{$\top_1$} % first child of binary inference
		\AxiomC{$\top_2$} % grandchild of binary inference
		\RightLabel{\scriptsize($\vee_2$)} % label of unary inference
		\UnaryInfC{$\bot \vee \top_2$} % second child of binary inference
		\RightLabel{\scriptsize($\wedge$)} % label of binary inference
		\BinaryInfC{$\top_1 \wedge (\bot \vee \top_2)$} % root
	\end{prooftree}	
	
	Note: it works like a stack.

\end{frame}

\begin{frame}[fragile]{More info}
	
	For more details, see corresponding manuals and guides:
	
	\begin{itemize}
		\item \LaTeX\ in general
		
		{\footnotesize Wiki: \verb|http://en.wikibooks.org/wiki/LaTeX|}
	
		\item Document class used: \verb|beamer|
		
		% Manual: http://www.tex.ac.uk/tex-archive/macros/latex/contrib/beamer/doc/beameruserguide.pdf
		{\footnotesize Tutorial: \verb|http://www.math.umbc.edu/~rouben/beamer/|}
		
		\item Verbatim environments: \verb|fancyvrb|
		
		{\footnotesize Manual: \verb|http://mirror.hmc.edu/ctan/macros/latex/contrib/fancyvrb/fancyvrb.pdf|}
		{\footnotesize Tutorial: \verb|http://code.haskell.org/SLPJ-collaborative-papers/styles/fancyvrb.pdf|}
		
		\item Proof trees: \verb|bussproofs|

		{\footnotesize Guide: \verb|http://www.logicmatters.net/resources/pdfs/latex/BussGuide2.pdf|}
	
	\end{itemize}
	
\end{frame}

\end{document}