\documentclass[10pt,a4paper]{beamer}
\usetheme{Singapore}
\setbeamercovered{transparent}
\beamertemplatenavigationsymbolsempty % Comment this out if you want a navigation bar

\usepackage[utf8]{inputenc}
\usepackage[english]{babel}

%math packages
\usepackage{amsmath}
\usepackage{amsfonts}
\usepackage{amssymb}

%package for rule inferences
\usepackage{bussproofs}

%package for handling verbatim
\usepackage{fancyvrb}

% packages for drawing
\usepackage{pgf}
\usepackage{tikz}
\usetikzlibrary{arrows,matrix}

% package for strikethrough text
\usepackage{ulem}

\logo{
	\includegraphics[height=1cm]{coq_logo.png}
}

% uncover everything in a step-wise fashion
%\beamerdefaultoverlayspecification{<+->}

\usepackage{color}
\usepackage{xcolor}

\definecolor{coqGreen}{rgb}{0,0.5,0}

% Environment for user input / source code
\DefineVerbatimEnvironment
{user}{Verbatim}
{formatcom=\color{coqGreen}}

% Environment for Coq output
\DefineVerbatimEnvironment
{coq}{Verbatim}
{formatcom=\color{blue}}

% Customizable emphasis
\providecommand{\highlight}[1]{\textcolor{red}{#1}}
\providecommand{\Highlight}[1]{\textcolor{red}{\textbf{#1}}}
\title{Coq Proof Assistant: Propositions and Proofs}
\author{Mirco Kocher}
\institute{
	Logic and Theory Group\\
	Institute of Computer Science and Applied Mathematics\\
	Universität Bern
}
\date{2012}
\begin{document}

{ \logo{} % Suppress logo for title page
\begin{frame} % Title page
	\begin{center}
		\includegraphics{coq_logo.png}
	\end{center}	
	\titlepage	
\end{frame}
}

\begin{frame}{Overview}
\tableofcontents[pausesections, pausesubsections]
\end{frame}

\section{Minimal Propositional Logic}

\subsection{Basics}
\begin{frame}{Truth table}
	$(P \rightarrow Q)$
	\begin{itemize}
		\item Classical logic
		\pause
		\item Assign to every variable a denotation true or false
		\pause		
		\item Formula is valid iff true in all cases
		\pause
		\item Question "is the proposition P true?"
	\end{itemize}
\end{frame}
\begin{frame}{Coq system}
	$(P \rightarrow Q)$
	\begin{itemize}
		\item Intuitionistic logic
		\pause
		\item Obtain a proof of Q from a proof of P
		\pause		
		\item Arbitrary proof of P constructs a proof of Q
		\pause
		\item Question "what are the proof of P (if any)?"
	\end{itemize}
\end{frame}

\subsection{Definition}
\begin{frame}[fragile]{Hypothesis}
	\begin{user}
	Hypothesis h:P
	\end{user}
	\begin{itemize}
		\item Local declaration
		\pause
		\item $h$ is the name of the hypothesis
		\pause
		\item P is its statement
		\pause
		\item Synonymous to
		\begin{user}
		Variable h:P
		\end{user}
		\pause
		\item Use
		\begin{user}
		Hypotheses
		\end{user}
		or
		\begin{user}
		Variables
		\end{user}
		to declare several at a time
	\end{itemize}
\end{frame}
\begin{frame}[fragile]{Section}
	The section contains all Hypoteses / Variables from the Context\\
	\pause
	\medskip
	Start section sec1 with
	\begin{user}
	Section sec1
	\end{user}
	\pause
	\medskip
	End section sec1 with
	\begin{user}
	End sec1
	\end{user}
\end{frame}
\begin{frame}[fragile]{Axiom}
	\begin{user}
	Axiom x:P
	\end{user}
	\begin{itemize}
		\item Global declaration
		\pause
		\item Synonymous to
		\begin{user}
		Parameter x:P
		\end{user}
	\end{itemize}
	\pause
	\medskip
	Environment contains axioms\\
	\pause
	\medskip
	Context contains hypotheses\\
	\pause
	\medskip
	$E, \Gamma \vdash \pi : P$
\end{frame}
\begin{frame}{Goals and Tactics}
	Goals: what needs to be proven\\
	\medskip
	\pause
	Goal: $E, \Gamma \vdash P$\\
	\pause
	Construct a proof of P. Should be a well-formed term $t$ in the environment $E$ and context $\Gamma$\\
	\pause
	Term $t$ is called a $solution$\\
	\pause
	\bigskip
	Tactics: commands to decompose this goal into simpler goals\\
	\pause
	\medskip
	$g$ ist input goal and $g_1$, $g_2$, ..., $g_k$ are output goals\\
	\pause
	Possible to construct a solution of $g$ from the solutions of goals $g_i$	
\end{frame}

\section{Example}

\subsection{Definition}
\begin{frame}[fragile]{intros}
	Goal: $(P \rightarrow Q) \rightarrow (Q \rightarrow R) \rightarrow (P \rightarrow R)$ \\
	\medskip
	\pause
	\begin{user}
	intros H H' p
	\end{user}
	\pause
	\begin{itemize}
		\item Transform the task of construction a proof into proving R with those hypotheses added
		\pause
		\item $H:P \rightarrow Q$
		\pause
		\item $H':Q \rightarrow R$ and
		\pause
		\item $p:P$
		\pause
		\item New subgoal: $R$
	\end{itemize}
	\pause
	Simplifies the statement to prove and increases the resources available
\end{frame}
\begin{frame}[fragile]{apply}
	Subgoal: $R$\\
	\medskip
	Hypothesis $H': Q \rightarrow R$ and $H:P \rightarrow Q$
	\medskip
	\pause
	\begin{user}
	apply H'
	\end{user}
	\begin{itemize}
		\item Use the hypothesis $H'$ to advance our proof
		\pause
		\item Argument has to be a premise and a conclusion
		\pause
		\item Creates new goal for the premise
		\pause
		\item New subgoal: $Q$
	\end{itemize}
	\pause
	Applying hypothesis H gives the new subgoal $P$
\end{frame}
\begin{frame}[fragile]{assumption}
	Subgoal: $P$\\
	\medskip
	Hypothesis $p: P$
	\medskip
	\pause
	\begin{user}
	assumption
	\end{user}
	\begin{itemize}
		\item Statement to proof is exactly statement of hypothesis $p$
		\pause
		\item Succeeds without generating any new goal
	\end{itemize}
	\pause
	\medskip
	\begin{coq}
	No more subgoals
	\end{coq}
	\pause
	Proof complete
\end{frame}
\begin{frame}[fragile]{Finish}
	\begin{user}
	Qed
	\end{user}
	\pause
	\begin{itemize}
		\item Saves the theorem's name, statement and proof term
		\pause
		\item Displays the sequence of tactics.
	\end{itemize}
	\begin{coq}
	intros H H' p.
	apply H'.
	apply H.
	assumption.
	\end{coq}
	\pause
	\medskip
	\begin{user}
	Print theorem-name
	\end{user}
	\pause
	\begin{itemize}
		\item Shows the proof like any $Gallina$ definition
	\end{itemize}
	\begin{coq}
	theorem-name = fun (H:P -> Q)(H':Q -> R)(p:P) => H' (H p)
		 : (P -> Q) -> (Q -> R) -> P -> R
	\end{coq}
\end{frame}

\subsection{Demo}
\begin{frame}{Transitivity}
	\begin{center}
		$(P \rightarrow Q) \rightarrow (Q \rightarrow R) \rightarrow (P \rightarrow R)$
	\end{center}
\end{frame}


%%%%%%%%%%%%%%%%

\section{Default}

\begin{frame}{Emphasis is everything}
	The following word is \highlight{emphasized} is a way that's \highlight{clearly visible} on a beamer.
	In case you want a \highlight{stronger} emphasis, it's \Highlight{possible too}.	
	\bigskip	
	Commands used for that are defined in preamble.tex, you can tweak the visual style from one place.
\end{frame}
\begin{frame}{Columns and paragraphs}
	\begin{columns}		
		\begin{column}{0.4\textwidth}		
			Arranging it in columns is also a possibility.			
		\end{column}			
		\begin{column}{0.6\textwidth}		
			Note that column width can be custom.				
		\end{column}				
	\end{columns}	
\end{frame}
\begin{frame}{Inference trees}
	You can use \highlight{bussproofs} to display inference rules and derivations:	
	\begin{prooftree}	
		\AxiomC{$\top_1$} % first child of binary inference
		\AxiomC{$\top_2$} % grandchild of binary inference
		\RightLabel{\scriptsize($\vee_2$)} % label of unary inference
		\UnaryInfC{$\bot \vee \top_2$} % second child of binary inference
		\RightLabel{\scriptsize($\wedge$)} % label of binary inference
		\BinaryInfC{$\top_1 \wedge (\bot \vee \top_2)$} % root
	\end{prooftree}	
	Note: it works like a stack.
\end{frame}

\end{document}